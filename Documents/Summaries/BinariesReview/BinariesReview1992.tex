\documentclass{article}
\usepackage[utf8]{inputenc}

\title{Summaries}
\author{Manuel Pichardo}
\date{February 2016}

\usepackage[framemethod=TikZ]{mdframed}
\newcounter{para}[subsubsection]


\newcommand{\para}{
\vspace{.3cm}
\addtocounter{para}{1}
\arabic{para}
\vspace{.3cm}\\
}




\mdfdefinestyle{MyFrame}{%
    linecolor=blue,
    outerlinewidth=2pt,
    roundcorner=20pt,
    innertopmargin=\baselineskip,
    innerbottommargin=\baselineskip,
    innerrightmargin=20pt,
    innerleftmargin=20pt,
    backgroundcolor=gray!50!white}

\begin{document}

\maketitle

\chapter{Binaries in Globular Clusters}

\section{Chapter 1}



\subsection{Chapter 3}

\subsubsection{Globular Cluster Evolution}

\vspace{.3cm}
\addtocounter{para}{1}
\arabic{para}
\vspace{.3cm}

Stable dynamical equilibrium but not in "thermal" time scale. Evaporation cause core collapse due to "gravothermal instability"



\begin{itemize}
    \item Evaporation of stars?
    \item gravothermal?
\end{itemize}


\vspace{.3cm}
\addtocounter{para}{1}
\arabic{para}
\vspace{.3cm}

Core collpase a few times larget than half-mass relazation time. Posible energy sources to avoid infinite core density:

\begin{enumerate}
    \item Increae bining enerygy binaries
    \item Star mass Loss
    \item Black HOle
\end{enumerate}
\begin{itemize}
    \item Half-mass realization time.
\end{itemize}



\subsubsection{Core Collapse}

\para
Dead by evapolation inevitable but  then shows core collpase happnes before. 

\begin{itemize}
    \item Negative Head capacity
\end{itemize}

\para
IN the 70 discovered X-rays from capture Neutron Stars. Core collpase seen M51. 

\subsubsection{Post-Collpase Evolution}

\para
Post-collpase and cluster evapolation. Gravothermal oscillations found

\para
seeing limited cores == sign of core collapse. 

\para
Core bounce. $r_h \propto t^{2/3}$ and for velocity dispersion $v \propto t^{-1/3}$. Where r is the hlaf-mass radius and t is time since bounce core. 

\para
derivation result above

\para
So regardes the half-mass radius expands steadely and as it expands the galaclactic tidal field removes the outermost stars. THe time is longer by only a few factos thatn core-collpase. 



\subsubsection{Central Enery Source}

\para
Three energy sourdes:1-Extracted Binding Energy, 2-Mass loss, 3Black Holes due to repeted merging. 

Binaries can be formed by capture. Mass loss can be by merging stars froming shor live ones. 

\para
All result in Heating. 
Binaries:

Hard Binary has binding energy $>> 1kt$  and ($3/2)kt$ is mean stellat kinetic energy. Hard binaries when interact with a thrid thend to equpartition and gives energy to single escaping star. So hard binaries tend to heat enviromment. 

\para
mass loss now. More indirect.  can losse stars or by winds, and supernovae. Loss due to vitial theorem. Loss mass by fraction $\epsilon$ and potential energy is quatraic so decreases by a factor $2\epsilon$. much more than kinetic enrgy so  the initial virial theorem configuration loss where U waas 2 of K. 

\para 
Black Holes
Caputre stars of central region. have relative small K. Capture tend to inrease relative temparaure of remaining Population. 

\subsubsection{Core OScillations}


\para
\textbf{"Gravothermal oscillitions"}
Due to decouplaing of both inner and outer radius. Increase more with star densities. Different time scales. 

\para
Confirmed maybe it matters binary population and other parameters. 

\para
Need simulation to verify

\subsubsection{Binary-Star Evolution}

\textbf{Physical Mechanism}

\para
Mechanism:
\begin{enumerate}
	\item Mass Segragation
	\item 3 body interaction
	\item binary-binary inte
	\item recoil and ejection
	\item collision and coaescence
	\item spiraling
	\item Stllar and binary evolution
\end{enumerate}

\begin{itemize}
	\item Diffrence spiraling and coalesnce. 
\end{itemize}


\para
Mass segragation since binary are heavier tend to go to core. Softer binaries tend to be "ionized" r destroyed in that trip to center and intercations enountered. Harder heat the cluster and harden. 

\para
It hardens and interaction less frequent and mmore violent. It net heating rate averaged over many relacation times is constants to be $0.3kTtT.$ whEher t is relaxation time. 

\para
Most near core and heating can be thought to be localized. 

\para
Binary-Binary have more outcomes. Most likely destroy the wider one and harden other. So two single and hard binary produced. Second most likely is ejection of one star and fome tripled. But not stable in dense medium like cluster. Efficient destyoing wide syste,

\para
Binary continues hardeing and recoiling and can escape cluster. Avoind by collision or spiral due to gravitational radiation


\para
Need Simulation

\subsubsection{Point Mass Dynamics}

\para
In dense clusters more than Eq. 8 binareis with Period longer thatn ---- will hava interacted with other stars so no "primordial" with larger Period. 

\para
MOnte-Carlo simulation with different binaries and impact parametrs

\para
HEavy mass stay in final binary. And change place with lighter (?) and wider binary?


\para
So heavier binarys and wider so bigger area of influence so more econounters
Binaries effective at sucking heavy stars. 

\subsubsection{Tidal Capture}

\para
Not that sensity to density but process still in core mainly. Only in really large dense ones this tidal binary-formation rate is significant. Maybe formed some bluestragles. Focus on X rays souces. 

\begin{itemize}
	\item Blue Stranglers (?)
\end{itemize}

. Focus on X rays souces. 








\end{document}
